% Options for packages loaded elsewhere
\PassOptionsToPackage{unicode}{hyperref}
\PassOptionsToPackage{hyphens}{url}
\PassOptionsToPackage{dvipsnames,svgnames,x11names}{xcolor}
%
\documentclass[
  letterpaper,
  DIV=11,
  numbers=noendperiod]{scrreprt}

\usepackage{amsmath,amssymb}
\usepackage{lmodern}
\usepackage{iftex}
\ifPDFTeX
  \usepackage[T1]{fontenc}
  \usepackage[utf8]{inputenc}
  \usepackage{textcomp} % provide euro and other symbols
\else % if luatex or xetex
  \usepackage{unicode-math}
  \defaultfontfeatures{Scale=MatchLowercase}
  \defaultfontfeatures[\rmfamily]{Ligatures=TeX,Scale=1}
\fi
% Use upquote if available, for straight quotes in verbatim environments
\IfFileExists{upquote.sty}{\usepackage{upquote}}{}
\IfFileExists{microtype.sty}{% use microtype if available
  \usepackage[]{microtype}
  \UseMicrotypeSet[protrusion]{basicmath} % disable protrusion for tt fonts
}{}
\makeatletter
\@ifundefined{KOMAClassName}{% if non-KOMA class
  \IfFileExists{parskip.sty}{%
    \usepackage{parskip}
  }{% else
    \setlength{\parindent}{0pt}
    \setlength{\parskip}{6pt plus 2pt minus 1pt}}
}{% if KOMA class
  \KOMAoptions{parskip=half}}
\makeatother
\usepackage{xcolor}
\setlength{\emergencystretch}{3em} % prevent overfull lines
\setcounter{secnumdepth}{-\maxdimen} % remove section numbering
% Make \paragraph and \subparagraph free-standing
\ifx\paragraph\undefined\else
  \let\oldparagraph\paragraph
  \renewcommand{\paragraph}[1]{\oldparagraph{#1}\mbox{}}
\fi
\ifx\subparagraph\undefined\else
  \let\oldsubparagraph\subparagraph
  \renewcommand{\subparagraph}[1]{\oldsubparagraph{#1}\mbox{}}
\fi


\providecommand{\tightlist}{%
  \setlength{\itemsep}{0pt}\setlength{\parskip}{0pt}}\usepackage{longtable,booktabs,array}
\usepackage{calc} % for calculating minipage widths
% Correct order of tables after \paragraph or \subparagraph
\usepackage{etoolbox}
\makeatletter
\patchcmd\longtable{\par}{\if@noskipsec\mbox{}\fi\par}{}{}
\makeatother
% Allow footnotes in longtable head/foot
\IfFileExists{footnotehyper.sty}{\usepackage{footnotehyper}}{\usepackage{footnote}}
\makesavenoteenv{longtable}
\usepackage{graphicx}
\makeatletter
\def\maxwidth{\ifdim\Gin@nat@width>\linewidth\linewidth\else\Gin@nat@width\fi}
\def\maxheight{\ifdim\Gin@nat@height>\textheight\textheight\else\Gin@nat@height\fi}
\makeatother
% Scale images if necessary, so that they will not overflow the page
% margins by default, and it is still possible to overwrite the defaults
% using explicit options in \includegraphics[width, height, ...]{}
\setkeys{Gin}{width=\maxwidth,height=\maxheight,keepaspectratio}
% Set default figure placement to htbp
\makeatletter
\def\fps@figure{htbp}
\makeatother

\KOMAoption{captions}{tableheading}
\makeatletter
\makeatother
\makeatletter
\makeatother
\makeatletter
\@ifpackageloaded{caption}{}{\usepackage{caption}}
\AtBeginDocument{%
\ifdefined\contentsname
  \renewcommand*\contentsname{Table of contents}
\else
  \newcommand\contentsname{Table of contents}
\fi
\ifdefined\listfigurename
  \renewcommand*\listfigurename{List of Figures}
\else
  \newcommand\listfigurename{List of Figures}
\fi
\ifdefined\listtablename
  \renewcommand*\listtablename{List of Tables}
\else
  \newcommand\listtablename{List of Tables}
\fi
\ifdefined\figurename
  \renewcommand*\figurename{Figure}
\else
  \newcommand\figurename{Figure}
\fi
\ifdefined\tablename
  \renewcommand*\tablename{Table}
\else
  \newcommand\tablename{Table}
\fi
}
\@ifpackageloaded{float}{}{\usepackage{float}}
\floatstyle{ruled}
\@ifundefined{c@chapter}{\newfloat{codelisting}{h}{lop}}{\newfloat{codelisting}{h}{lop}[chapter]}
\floatname{codelisting}{Listing}
\newcommand*\listoflistings{\listof{codelisting}{List of Listings}}
\makeatother
\makeatletter
\@ifpackageloaded{caption}{}{\usepackage{caption}}
\@ifpackageloaded{subcaption}{}{\usepackage{subcaption}}
\makeatother
\makeatletter
\@ifpackageloaded{tcolorbox}{}{\usepackage[many]{tcolorbox}}
\makeatother
\makeatletter
\@ifundefined{shadecolor}{\definecolor{shadecolor}{rgb}{.97, .97, .97}}
\makeatother
\makeatletter
\makeatother
\ifLuaTeX
  \usepackage{selnolig}  % disable illegal ligatures
\fi
\IfFileExists{bookmark.sty}{\usepackage{bookmark}}{\usepackage{hyperref}}
\IfFileExists{xurl.sty}{\usepackage{xurl}}{} % add URL line breaks if available
\urlstyle{same} % disable monospaced font for URLs
\hypersetup{
  colorlinks=true,
  linkcolor={blue},
  filecolor={Maroon},
  citecolor={Blue},
  urlcolor={Blue},
  pdfcreator={LaTeX via pandoc}}

\author{}
\date{}

\begin{document}
\ifdefined\Shaded\renewenvironment{Shaded}{\begin{tcolorbox}[frame hidden, sharp corners, enhanced, borderline west={3pt}{0pt}{shadecolor}, interior hidden, boxrule=0pt, breakable]}{\end{tcolorbox}}\fi

\hypertarget{Principles}{%
\chapter*{Principles~}\label{Principles}}
\addcontentsline{toc}{chapter}{Principles~}

~

These guidelines will support staff of the School of Psychology, Queen's
University Belfast, in their efforts to engage in Open Science
practices, now considered to be at the cutting edge of our discipline.
We aim to inspire staff to engage with Open Science; we should all
aspire to adopt Open Science practices. Over time, this may well become
requirements (e.g., from publishers, funders, REF), but it makes sense
to anticipate this. These guidelines and Open Science in general are
governed by several principles.~

\hypertarget{kindness}{%
\section{Kindness~}\label{kindness}}

These Open Science guidelines are written as an accessible guide to Open
Science for academic and research staff in the School of Psychology.
However, they should also be helpful for professional service staff and
students. In the spirit of Open Science, the guidelines are available
for all including those from outside the institution. The following
guide is a living document, which will be continually updated to keep
the information aligned with advances in Open Science, so it is best to
refer to the most current version of the guidelines.~

\hypertarget{equity-and-diversity}{%
\section{Equity and Diversity~}\label{equity-and-diversity}}

Open Science emphasises transparency and integrity in accurate and
truthful reporting of science. Being open and transparent at all stages
and all levels of the scientific process improves all scientific fields.
The principles and proposed practices of Open Science serve to expand
the accessibility and inclusivity of research and the knowledge it
creates. By making data, analysis methods, and other research materials
available to all through online repositories, many of the barriers to
participation in science for those outside of elite academic
institutions are removed. In this sense, Open Science principles are
aligned to enhancing equity in science and expanding the diversity of
future research as a result.~

\hypertarget{why-is-open-science-important}{%
\section{Why is Open Science
important?~}\label{why-is-open-science-important}}

Open Science directly addresses the `replication crisis' initially
identified in the field of Psychology. The replication crisis happened
because scientists~ were incentivised (by the academic reward structure)
to report results that often failed to replicate (e.g., `false
positives'). Open Science is a mindset to do what is right for science,
to strive for~~

- transparency of motivations, methods and inferences~~

- reproducibility of scientific methods~~

- robustness of patterns (i.e., replicability)~

Open Science emphasises transparency and integrity in accurate and
truthful reporting of science.\,Being open and transparent at all stages
and all levels of the scientific process improves all scientific
fields.~~

There is an increasing recognition of the value of Open Science
practices proposed to fulfil these aims. Indeed, engagement with Open
Science is being embedded in criteria for 1) open access publishing in
high impact journals, 2) research funding, especially by national and
international funders 3) the Research Excellent Framework, 4) academic
job descriptions and progression and 5) Higher Education teaching
curricula.~

\hypertarget{open-science-practices}{%
\section{Open Science Practices~}\label{open-science-practices}}

To help embed Open Science within the research culture of the School of
Psychology, this guide will provide descriptions and practical advice on
key Open Science practices. Section 1 will cover Preregistration;
Section 2 Ethics and Open Science; Section 3 Open data/materials;
Section 4 Open-Access Publishing and Section 5 Education.~

\hypertarget{feedback-and-additional-information}{%
\section{Feedback and additional
information~}\label{feedback-and-additional-information}}

This document is written by your colleagues to support the School's
efforts to be more engaged in open science practices. If you spot any
errors, would like to add any amendments, or have an innovation or
efficient practice to share, please fill out the webform:~~

If you require any advice or support regarding your Open Science
practice feel free to contact the authors at~

Joost Dessing
\href{mailto:j.dessing@qub.ac.uk}{\nolinkurl{j.dessing@qub.ac.uk}}~

Lisa Graham-Wisener
\href{mailto:l.grahamwisener@qub.ac.uk}{\nolinkurl{l.grahamwisener@qub.ac.uk}}~~

Gary McKeown
\href{mailto:g.mckeown@qub.ac.uk}{\nolinkurl{g.mckeown@qub.ac.uk}}~~

Paul Toner
\href{mailto:p.toner@qub.ac.uk}{\nolinkurl{p.toner@qub.ac.uk}}~~

Matthew Rodger \href{mailto:m.rodgers@qub.ac.uk}{m.rodger@qub.ac.uk}~

Gillian Shorter
\href{mailto:g.shorter@qub.ac.uk}{\nolinkurl{g.shorter@qub.ac.uk}}~

Tanja Gerlach
\href{mailto:t.gerlach@qub.ac.uk}{\nolinkurl{t.gerlach@qub.ac.uk}}~

Thomas Schultze-Gerlach
\href{mailto:t.schultze@qub.ac.uk}{\nolinkurl{t.schultze@qub.ac.uk}}~



\end{document}
