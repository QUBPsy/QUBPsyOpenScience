% Options for packages loaded elsewhere
\PassOptionsToPackage{unicode}{hyperref}
\PassOptionsToPackage{hyphens}{url}
\PassOptionsToPackage{dvipsnames,svgnames,x11names}{xcolor}
%
\documentclass[
  letterpaper,
  DIV=11,
  numbers=noendperiod]{scrreprt}

\usepackage{amsmath,amssymb}
\usepackage{lmodern}
\usepackage{iftex}
\ifPDFTeX
  \usepackage[T1]{fontenc}
  \usepackage[utf8]{inputenc}
  \usepackage{textcomp} % provide euro and other symbols
\else % if luatex or xetex
  \usepackage{unicode-math}
  \defaultfontfeatures{Scale=MatchLowercase}
  \defaultfontfeatures[\rmfamily]{Ligatures=TeX,Scale=1}
\fi
% Use upquote if available, for straight quotes in verbatim environments
\IfFileExists{upquote.sty}{\usepackage{upquote}}{}
\IfFileExists{microtype.sty}{% use microtype if available
  \usepackage[]{microtype}
  \UseMicrotypeSet[protrusion]{basicmath} % disable protrusion for tt fonts
}{}
\makeatletter
\@ifundefined{KOMAClassName}{% if non-KOMA class
  \IfFileExists{parskip.sty}{%
    \usepackage{parskip}
  }{% else
    \setlength{\parindent}{0pt}
    \setlength{\parskip}{6pt plus 2pt minus 1pt}}
}{% if KOMA class
  \KOMAoptions{parskip=half}}
\makeatother
\usepackage{xcolor}
\setlength{\emergencystretch}{3em} % prevent overfull lines
\setcounter{secnumdepth}{-\maxdimen} % remove section numbering
% Make \paragraph and \subparagraph free-standing
\ifx\paragraph\undefined\else
  \let\oldparagraph\paragraph
  \renewcommand{\paragraph}[1]{\oldparagraph{#1}\mbox{}}
\fi
\ifx\subparagraph\undefined\else
  \let\oldsubparagraph\subparagraph
  \renewcommand{\subparagraph}[1]{\oldsubparagraph{#1}\mbox{}}
\fi


\providecommand{\tightlist}{%
  \setlength{\itemsep}{0pt}\setlength{\parskip}{0pt}}\usepackage{longtable,booktabs,array}
\usepackage{calc} % for calculating minipage widths
% Correct order of tables after \paragraph or \subparagraph
\usepackage{etoolbox}
\makeatletter
\patchcmd\longtable{\par}{\if@noskipsec\mbox{}\fi\par}{}{}
\makeatother
% Allow footnotes in longtable head/foot
\IfFileExists{footnotehyper.sty}{\usepackage{footnotehyper}}{\usepackage{footnote}}
\makesavenoteenv{longtable}
\usepackage{graphicx}
\makeatletter
\def\maxwidth{\ifdim\Gin@nat@width>\linewidth\linewidth\else\Gin@nat@width\fi}
\def\maxheight{\ifdim\Gin@nat@height>\textheight\textheight\else\Gin@nat@height\fi}
\makeatother
% Scale images if necessary, so that they will not overflow the page
% margins by default, and it is still possible to overwrite the defaults
% using explicit options in \includegraphics[width, height, ...]{}
\setkeys{Gin}{width=\maxwidth,height=\maxheight,keepaspectratio}
% Set default figure placement to htbp
\makeatletter
\def\fps@figure{htbp}
\makeatother

\KOMAoption{captions}{tableheading}
\makeatletter
\makeatother
\makeatletter
\makeatother
\makeatletter
\@ifpackageloaded{caption}{}{\usepackage{caption}}
\AtBeginDocument{%
\ifdefined\contentsname
  \renewcommand*\contentsname{Table of contents}
\else
  \newcommand\contentsname{Table of contents}
\fi
\ifdefined\listfigurename
  \renewcommand*\listfigurename{List of Figures}
\else
  \newcommand\listfigurename{List of Figures}
\fi
\ifdefined\listtablename
  \renewcommand*\listtablename{List of Tables}
\else
  \newcommand\listtablename{List of Tables}
\fi
\ifdefined\figurename
  \renewcommand*\figurename{Figure}
\else
  \newcommand\figurename{Figure}
\fi
\ifdefined\tablename
  \renewcommand*\tablename{Table}
\else
  \newcommand\tablename{Table}
\fi
}
\@ifpackageloaded{float}{}{\usepackage{float}}
\floatstyle{ruled}
\@ifundefined{c@chapter}{\newfloat{codelisting}{h}{lop}}{\newfloat{codelisting}{h}{lop}[chapter]}
\floatname{codelisting}{Listing}
\newcommand*\listoflistings{\listof{codelisting}{List of Listings}}
\makeatother
\makeatletter
\@ifpackageloaded{caption}{}{\usepackage{caption}}
\@ifpackageloaded{subcaption}{}{\usepackage{subcaption}}
\makeatother
\makeatletter
\@ifpackageloaded{tcolorbox}{}{\usepackage[many]{tcolorbox}}
\makeatother
\makeatletter
\@ifundefined{shadecolor}{\definecolor{shadecolor}{rgb}{.97, .97, .97}}
\makeatother
\makeatletter
\makeatother
\ifLuaTeX
  \usepackage{selnolig}  % disable illegal ligatures
\fi
\IfFileExists{bookmark.sty}{\usepackage{bookmark}}{\usepackage{hyperref}}
\IfFileExists{xurl.sty}{\usepackage{xurl}}{} % add URL line breaks if available
\urlstyle{same} % disable monospaced font for URLs
\hypersetup{
  colorlinks=true,
  linkcolor={blue},
  filecolor={Maroon},
  citecolor={Blue},
  urlcolor={Blue},
  pdfcreator={LaTeX via pandoc}}

\author{}
\date{}

\begin{document}
\ifdefined\Shaded\renewenvironment{Shaded}{\begin{tcolorbox}[frame hidden, boxrule=0pt, breakable, sharp corners, enhanced, interior hidden, borderline west={3pt}{0pt}{shadecolor}]}{\end{tcolorbox}}\fi

\hypertarget{Preregistration}{%
\chapter{Preregistration}\label{Preregistration}}

\textbf{We encourage staff to preregister their research as standard.}~

\hypertarget{what-is-preregistration}{%
\section{What is preregistration?~}\label{what-is-preregistration}}

\href{https://www.pnas.org/doi/10.1073/pnas.1708274114}{Preregistration}
involves publishing your plans in a protocol for research before you
conduct the research. This preregistration should be version-controlled,
time-stamped, and deposited in a publicly accessible repository. It
ensures that any change to the research plans or protocol can be traced,
which disincentivizes presenting a changed protocol as the original,
intended protocol. The more detailed the preregistered protocol, the
more preregistration prevents results being influenced -- even
inadvertently -- by undisclosed flexibility in research practices, which
is known to affect replicability. Most pre-registered information should
include the following:~

\begin{enumerate}
\def\labelenumi{\arabic{enumi}.}
\tightlist
\item
  A statement of the hypotheses/research questions being assessed~
\end{enumerate}

\begin{enumerate}
\def\labelenumi{\arabic{enumi}.}
\setcounter{enumi}{1}
\tightlist
\item
  A clear methodology which can be replicated by other researchers~
\end{enumerate}

\begin{enumerate}
\def\labelenumi{\arabic{enumi}.}
\setcounter{enumi}{2}
\tightlist
\item
  The key variables that will be measured and/or theoretical lens by
  which information will be viewed~
\end{enumerate}

\begin{enumerate}
\def\labelenumi{\arabic{enumi}.}
\setcounter{enumi}{3}
\tightlist
\item
  An indication of the intended sample size (ideally justified with a
  power analysis for quantitative research) and how this will be
  obtained (recruitment/retention plans)~
\end{enumerate}

\begin{enumerate}
\def\labelenumi{\arabic{enumi}.}
\setcounter{enumi}{4}
\tightlist
\item
  A detailed description of the analysis that will be conducted~
\end{enumerate}

\hypertarget{what-are-the-benefits-of-preregistration}{%
\section{What are the benefits of
preregistration?~}\label{what-are-the-benefits-of-preregistration}}

\begin{itemize}
\item
  More robust and transparent research: increased trustworthiness and
  confidence in replicability~~
\item
  Clarity on dissociation between confirmatory and exploratory
  analyses~~~
\item
  Clarity on plan for research team -- a document to guide data
  analyses~~
\end{itemize}

\begin{itemize}
\item
  Avoiding any suggestion of benefiting from undisclosed flexibility~~
\item
  Available as article type (Registered Reports), ensuring publication
  of confirmatory science irrespective of results~~
\item
  Researchers are being criticized for P-hacking (reporting only
  significant findings) and
  \href{https://journals.sagepub.com/doi/10.1207/s15327957pspr0203_4}{HARKing}
  (hypothesising after the results are known) within scientific
  research; through preregistration we can illustrate that we do not
  engage in these practices~
\end{itemize}

\hypertarget{what-types-of-research-could-be-preregistered}{%
\section{What types of research could be
preregistered?~}\label{what-types-of-research-could-be-preregistered}}

All types of research can be pre-registered. Preregistration can be used
for both confirmatory and exploratory research, in any research approach
-- quantitative and qualitative. Besides quantitative experimental
designs, this includes among others observational designs, survey
designs, randomized trials, secondary data analyses, qualitative
interviews, ethnographies or focus groups, and systematic reviews. The
key here is that the preregistration provides transparency about what
the research plan was prior to data collection. For some designs such as
ethnography or qualitative interviews, minimal information may be known
in advance; for others such as randomised trials, much more information
will be known and available for preregistration.~

\hypertarget{how-do-i-preregister-my-plansprotocols-for-research}{%
\section{How do I preregister my plans/protocols for
research?~}\label{how-do-i-preregister-my-plansprotocols-for-research}}

How and where to register your protocols depends on the purpose and the
type of research you are conducting. Several online repositories have
been created that are optimized for the practice of preregistration and
transparency (e.g., version control, links to data). These are given
below.~

There may be different requirements for each type of research and the
\href{https://www.equator-network.org/}{Equator Network} have provided
\href{https://www.equator-network.org/?post_type=eq_guidelines\&eq_guidelines_study_design=study-protocols\&eq_guidelines_clinical_specialty=0\&eq_guidelines_report_section=0\&s=+}{guidance}
on what to include in protocols for a selection of different designs. To
date, this includes trials, intervention detail, systematic reviews,
tomography studies, intervention development studies, and core outcome
sets. You can also publish the protocol as a peer-reviewed journal
article, but this should be considered in addition to the registrations
below given the time the peer-review process takes. If your chosen
journal facilitates this (see
\href{https://v2.sherpa.ac.uk/romeo/}{Sherpa Romeo} you could deposit
the pre-print (i.e., the manuscript submitted for publication) at a
pre-print server such as \href{https://psyarxiv.com/}{PsyArXiv}.~

\hypertarget{main-pre-registration-repositories}{%
\section{Main pre-registration
repositories~}\label{main-pre-registration-repositories}}

Below, we list the main repositories to help staff get started with
preregistration on these portals.~

\href{https://aspredicted.org/}{\textbf{https://AsPredicted.org}}~

\begin{itemize}
\item
  Mainly for quantitative research~
\item
  Answer 10 questions about research protocol
\item
  Answers saved as pdf website~
\item
  Example: \href{https://aspredicted.org/p6956.pdf}{Valkanidis, Doumas,
  and Dessing (2020)}~
\item
  Researchers control the visibility (private or public).~
\end{itemize}

\href{https://osf.io/prereg/}{\textbf{Open Science Framework \textbar{}
OSF Preregistration}}~

\begin{itemize}
\tightlist
\item
  Platform dedicated to Open Science.
\end{itemize}

\begin{itemize}
\item
  Variety of formats for specific types of research (e.g., quantitative,
  qualitative, systematic reviews, scoping reviews, secondary data
  analyses)~
\item
  Can be helpful for student preregistrations (supervisors are
  encouraged to create a single page for a given academic year cohort
  e.g., 2022/2023, with folders for each registered project)~
\item
  Examples: \href{https://osf.io/sqh36}{Reid and Dessing (2016)},
  \href{https://osf.io/xvhgg/}{Schultze, Gerlach and Rittich ( 2017)}
\end{itemize}

\href{https://clinicaltrials.gov/}{\textbf{ClinicalTrials.gov}}~

\begin{itemize}
\item
  Preregistration of clinical trials~
\item
  It is an ethical requirement as per the
  \href{http://www.wma.net/en/30publications/10policies/b3/}{Declaration
  of Helsinki} that every clinical trial must be registered in a
  publicly accessible database before recruitment of the first
  participant. A list of other registration sites are:
  \href{https://www.who.int/ictrp/network/primary/en/}{Primary
  Registries in the WHO Registry Network} or an
  \href{http://www.icmje.org/about-icmje/faqs/clinical-trials-registration/}{ICMJE
  approved registry}. Please note that some of these may incur a charge.
  For trials you should also complete a
  \href{https://www.equator-network.org/reporting-guidelines/spirit-2013-statement-defining-standard-protocol-items-for-clinical-trials/}{SPIRIT
  protocol} document for your Trial Master file.~
\end{itemize}

\href{https://www.crd.york.ac.uk/prospero/}{\textbf{PROSPERO
(york.ac.uk)}}~

\begin{itemize}
\tightlist
\item
  Preregistration of systematic reviews~
\end{itemize}

\begin{itemize}
\tightlist
\item
  If you are aiming to publish your systematic review, you may severely
  limit your publication chances if you do not preregister your review.
  Good guidance on systematic review conduct can be found on
  \href{https://www.york.ac.uk/media/crd/Systematic_Reviews.pdf}{York
  Centre for Reviews and Dissemination}. To minimize work for the team
  at PROSPERO, only those reviews you intend to publish should be hosted
  at this site. For reviews in the context of teaching the Open Science
  Framework is more appropriate, but the preregistration could follow
  the structure of the PROSPERO registration.~~
\end{itemize}

\href{https://www.comet-initiative.org/}{\textbf{COMET Initiative}}~

\begin{itemize}
\tightlist
\item
  For core outcome sets or minimum data standards~
\end{itemize}

Please consider depositing your preregistration on the PURE platform
after you have registered it on one of the platforms above. To do so,
log into your PURE account, go to research outputs, other contribution,
protocol, and select either peer-reviewed (e.g., Prospero) or
non-peer-reviewed (e.g., OSF) and link to the DOI. You might also want
to highlight this as a protocol in the title of the deposit to minimise
confusion if you use the same title for the paper arising from the
work.~

\hypertarget{registered-reports}{%
\section{Registered reports~}\label{registered-reports}}

This is a form of preregistration, where a paper is accepted by journals
at the protocol stage, prior to data collection. The protocol goes
through the peer review process where the introduction, hypotheses,
methods and analysis plan are scrutinized before data is collected. This
is used for both quantitative and qualitative studies. If accepted, the
journal commits to publishing the full report irrespective of the
results, as long as the preregistered protocol is followed and the
conclusions are justified by the data.~

Once data is collected and analyzed the registered report is updated
with the findings and write up and is resubmitted to the same journal
for further peer review.~ You can read more about registered reports
\href{https://cos.io/rr/}{here}. The site linked also contains a list of
journals already offering the registered reports format. Note, however,
that this approach may have implications which compete with time demands
of external funders or student supervision.~

\hypertarget{common-questions}{%
\section{Common questions~}\label{common-questions}}

\emph{Doesn't preregistration stifle exploration and creativity?}~

No.~Nothing prevents further exploration of the patterns within a
dataset. Preregistration just ensures that such analyses are
appropriately identified as exploratory. Plans can also change, and this
can be reflected in an updated version of the preregistered protocol
prior to research start, or if research has already started, then
transparently reported and explained as a deviation from the protocol in
the write up.~

\emph{Is it extra work?}~

Yes and no. Yes, preregistration typically involves an in-depth research
protocol, the details of which may take more time to iron out. However,
since most research within Psychology requires an ethics application,
which also involves a research protocol, at least part of the
information is already written down even without preregistration.
Moreover, preregistration changes the order in which researchers spend
time: putting in more effort before data collection reduces the time
needed from data collection to dissemination.~

\emph{If I preregister elsewhere why do I need to add it to my PURE
repository?}~

Any preregistration on any site/portal implies that you are doing the
work; it creates an easily searchable protocol, which individuals who
are considering similar work can see to avoid duplication of effort
(e.g., in reviews), or to guide replication efforts (e.g., replication
studies).~

It is useful to deposit your published preregistration on PURE as it can
link to the final publication when it has been published; this
demonstrates your (and your co-authors') commitment to open science, and
institutional commitment to open science which grows our research
reputation.~



\end{document}
